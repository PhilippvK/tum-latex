% % % % % % % % % % % % % % % % % % % % % % % % % % % % % % % % % % % % % % % % 
% LaTeX4EI Template for Cheat Sheets                                Version 1.0
%					
% Authors: Emanuel Regnath, Martin Zellner
% Contact: info@latex4ei.de
% Encode: UTF-8, tabwidth = 4, newline = LF	
% % % % % % % % % % % % % % % % % % % % % % % % % % % % % % % % % % % % % % % % 


% ======================================================================
% Document Settings
% ======================================================================

% possible options: color/nocolor, english/german, threecolumn
% defaults: color, english
\documentclass[german]{latex4ei/latex4ei_sheet}

% set document information
\title{Stochastische\\Signale}
\author{Philipp van Kempen}					% optional, delete if unchanged
\myemail{philipp.van-kempen@tum.de}			% optional, delete if unchanged
\mywebsite{phi.philonata.de}			% optional, delete if unchanged


% ======================================================================
% Begin
% ======================================================================
\begin{document}


% Title
% ----------------------------------------------------------------------
\maketitle   % requires ./img/Logo.pdf


% Section
% ----------------------------------------------------------------------
\section*{Sonstiges}
\subsection*{Begriffe}
	\begin{itemize}
		\item Bitfolge: Entweder 0 oder 1
		\item Codew\"orter: $2^n$ M\"oglichkeiten
		\item Diskretes Experiment: Abz\"ahlbar viele Ergebnisse
		\item Deterministische Signale:?
		\item geordnete Paare/Tupel: $(a,b)$\\
			$(i,j)\neq(j,i), i\neq j$\\
			Gleichheit komponentenweise!
		\item ungeordnete Paare/Tupel: $\{a,b\}$\\
			$\{i,j\}=\{j,i\} \forall i,j$
		\item abz\"ahlbar unendlich:\\
		\item nicht abz\"ahlbar?
	\end{itemize}
\subsection*{Beispiele:}
\begin{itemize}
	\item $\Omega=\{(h1,h2,\ldots,h_6) \in \N_0^6 | h_1+ \cdots +h_6=2 \}$: TODO
	\item $A=\{(\omega_1,\omega_2) \in \Omega | \omega_1 + \omega_2 >9\}$: TODO
	\item $A'=\{\omega' \in \Omega' | \omega'>9\}$
	\item Kiosk: Hoher Eisabsatz, ist keine Ursache f\"ur sonniges Wetter, sondern eine Folge davon. ($P(S|E)\neq P(E|S)$)
	\item Ereignisse $A,B,C$:
		\begin{itemize}
			\item mindestens eins der Ereignisse: $A \cup B \cup C$
			\item h\"ochstens eins der Ereignisse: ${((A\ cap B) \cup (A \cap C) \cup (B \cap C))}^{c}$
			\item keins: $(A^c\cap B^c \cap C^c)={(A \cup B \cup C)}^c$
		\end{itemize}
	\item Beim Finden von disjunkten Teilmengen von sich schneidenden Mengen muss die Schnittmenge selbst auch als Menge aufgef\"uhrt werden.
	\item Gr\"unde f\"ur detailliertere Modellierung: sp\"ater doch auf Informationen zur\"uckgreifen, Modellbildung anhand von Nachfrage, was \"uberhaupt erwartet werden kann ohne selbst zu Messen was ankommt
	\item Don't Care: $(0,*)$
	\item Falls Wahrscheinlichkeitsmass zu pr\"ufen, gilt es Widerspr\"uche z. B. bei Teilmengen zu finden!
	\item Aus Mentorgruppe: $P(A \setminus B)=P(A)-P(A \cap B)$
\end{itemize}
\subsection*{Pr\"ufung}
Eiheiten abgeben, falls nicht normiert!\\
Achtung bei Abbildungen in anderen Wahrscheinlichkeitsraum! (Nicht immer 1 zu 1,$\Omega$ und $\Omega'$ \"aquivalent aber nicht gleich)\\
Empfehlenswert, verwendete K\"urzel definieren: $A,B \rightarrow $ Urne, $S \rightarrow $ S,$\dots$
\section{Einf\"uhrung}
\subsection{Zufall:} Etwas l\"asst sich nicht beschreiben oder man will es nicht beschreiben.\\
Zuf\"alligkeit $\neg$ Willk\"uhrlichkeit!\\
\subsection{Wahrscheinlichkeit:}
\textbf{Anwendbarkeit historischer Wahrscheinlichkeitsbegriffe:}\\
Verh\"altnisbegriff nur anwendbar, falls alle Ereignisse gleich wahrscheinlich.\\
Anzahl nur anwendbar falls Experiment wiederholbar\\
z. B. $P(''1'')=\frac{\# 1}{\# gesamt}$
\section{Wahrscheinlichkeitsr\"aume}

\[(\Omega,\F,P)\]

\subsection{Ergebnisraum $\Omega$}
Abh\"angig von Anwendung, vom Beobachtung bestimmt.\\
Menge von Beobachtungen, die wir bez\'uglich des Experiments machen k\"onnen. Jedes Ergebnis muss Teil des Ergebnisraumes sein.\\
Beispiele f\'ur Bitfolge $b_1b_2b_3, b\ in {0,1}$\\
$\Omega={(0,0,0),(0,0,1),\ldots,(1,1,1)}$ aber auch $\Omega={0,1,2,3}$\\
\textbf{M\"achtigkeit:} $|\Omega|=\#	$ Elemente\\
Ergebnismenge kann unendlich sein (Kopf/Zahl)
\subsection{$\sigma$-Algebra / Ereignisalgebra}
Minimaler Satz an in sich konsistenten Beobachtungen\\
Elementarereingis: Ereignis, besteht nur aus einem einzigen Ergebnis\\
Ereignisse $A,B \subset \Omega$\\
Gegenereignisse $A^c,B^c$\\
mit Komplement $A^c=\Omega \setminus A$\\
Differenzmenge $A \setminus B=$?\\
Schnittmenge $A \cap B, {(A \cap B)}^c$\\
Un\'ogliches Ereignis $\emptyset$\\
Sicheres Ereignis $\Omega$\\
Ergebnis $\neq$ Ereignis, da Ereignis Menge von Ergebnissen ist.\\
$\F \subset \P(\Omega)$ wobei Potenzmenge $\P(\Omega)$ Menge aller Teilmengen von $\Omega^2$ mit $|\P|=2^{|\Omega|}$\\
Kleinste ($\F=\{\emptyset,\Omega\}$) und gr\"o\ss{}te $\P(\Omega)$ $\sigma$-Algebra\\
Wenn sich $A$ und $B$ bestimmen lassen, kann man auch die Schnittmenge bestimmen.\\
$\#$ M\"oglichkeiten Ereignisse zu definieren: $2^{|\Omega|}$\\
\"Aquivalente Beschreibungen eines Ereignisses sind NICHT gleich weil sie auf anderen Ereignismengen basieren.
\textbf{Venn-Diagramm:}
TODO mit Identit\"aten aus Skript\\
$A,B$ partitionieren $\Omega$, sodass $G_1 \cup G_2 \cup \ldots=\Omega$ und $G_i \cap G_j = \emptyset , i \neq j$\\
M\"oglichkeiten:\\
\begin{itemize}
	\item $A,B$ schneiden sich: GRAFIK\\
		$G_1=A \setminus B$\\
		$G_2=B \setminus A$\\
		$G_3=A \cap B$\\
		$G_4=\Omega \setminus (A \cup B)$\\
		$|\F|=2^4$\\
		$\F=\{\emptyset,\Omega,A,A^c,B,B^c,A\cup B, {(A \,cup B)}^c,A\cap B,{(A \cap B)}^c\,A\setminus B,{(A\ setminus B)}^c,B\setminus A,{(B \setminus A)}^c,(A\setminus B) \cup (B \setminus A),{((A \setminus B) \cup (B \setminus A))}^c\}$
	\item $A,B$ disjunkt: GRAFIK\\
		$G_1=A$\\
		$G_2=B$\\
	    $G_3=\Omega \setminus (A \cup B)$\\
		$|\F|=2^3$\\
		$\F=\{\emptyset,\Omega,A,B,A^c,B^c,A \cup B,{(A \cup B)}^c\}$
		Aus disjunkten Teilmengen (Partitionierungen) kann man einfach eine gr\"o\ss{}ere Menge bauen
	\item $A$ alleine: GRAFIK\\
		$G_1=A$\\
		$G_2=\Omega \setminus A$\\
		$\F=\{\emptyset,\Omega,A,A^c\}$
\end{itemize}
\textbf{Bedingungen f\"ur Mengensystem $\F$ als $\sigma-$Algebra:}\\
\begin{enumerate}
	\item $\Omega \in \F$ ($\F$ nicht leer)
	\item $A \in \F \Rightarrow A^c \in \F$ ($\emptyset \in \F$)
	\item $A_1,A_2,\ldots \in F \Rightarrow \bigcup\limits_{i=1}^\infty A_i \in \F$\\
		mit De Morgan:\\
		\[\bigcap\limits_{i=1}^\infty A_i={(\bigcup\limits_{i=1}^\infty A_i^c)}^c\]
		sowie: $A_i \setminus A_j={(A_i^c \cup A_j)}^c$\\
		gilt: $\bigcap\limits_{i=1}^\infty A_i \in \F$ und $A_i \setminus A_j \in F$\\
\end{enumerate}
\textbf{Folgerung:} Eine $\sigma$-Algebra ist ein Mengensystem, welches gegen\"uber Komplementbildung, endlichen und abz\"ahlbaren Vereinigungen und Durchschnitten sowie gegen\"uber Mengendifferenz abgeschlossen ist.\\
Anzahl der Elemente muss immer gerade sein, da zu jedem Ereignis ein Gegenereignis existieren muss!\\
\subsubsection{Messraum}
oder auch \textbf{messbarer Raum}\\
\[(\Omega,\F)\]

\subsection{Wahrscheinlichkeitsma\ss{}$P$}
Man kann Wahrscheinlichkeiten nur Mengen, zuordnen, nicht den Elementen selbst.
\subsubsection{Axiome}
$A$: Ereignis
\begin{enumerate}
	\item Nichtnegativit\"at: $P(A)\ge 0$
	\item Normiertheit: $P(\Omega)=1$
	\item Additivit\"atssatz: $A_i \cap A_j=\emptyset \forall i \neq j$
		\[P(\bigcup\limits_{i=1}^\infty A_i)=\sum_{i=1}^\infty P(A_i)\]
\end{enumerate}
\textbf{Wahrscheinlichkeitsma\ss{}auf dem Messraum $(\Omega,\F)$:}\\
$P: \F \to [0,1]$\\
\subsubsection{Elementare Eigenschaften}
\begin{enumerate}
	\item $P(A^c)=1-P(A)$
	\item $P(\emptyset)=0$
	\item $P(A \setminus B)=P(A \cap B^c)=P(A)-P(A\cap B)$
	\item $P(A \cup B)=P(A)+P(B)-P(A\cap B)$
	\item $P(A \cap B)=P(A)+P(B)-P(A \cup B)$
	\item $A \subset B \Rightarrow P(A) \le P(B)$
	\item Union Bound:\\
		\[P(\bigcup\limits_{i=1}^k A_i) \le \sum_{i=1}^k P(A_i)\]
\end{enumerate}
\subsubsection{Interbretationen}
\begin{enumerate}
	\item TODO
\end{enumerate}
\subsection{Messbarkeit}
Probleme:
\begin{itemize}
	\item $\Omega$ nicht abz\"ahlbar: Beschr\"ankung auf Teilmengen n\"otig
\end{itemize}
\subsubsection{Borelsche $\sigma$-Algebra}
\[\B^n \overset{\Delta}{=} \B(\R^n)\]
$\sigma(G)=$kleinste $\sigma$-Algebra, die $G$ enth\"alt\\
Borelmengen $\B^n$\\
Erzeugendensystem:\\
\[G=\{\prod_{i=1}^n [a_i,b_i] \in \R^n | \forall a_i\le b_i , a_i,b_i \in Q\}\]
(Alle $n$-dimensionalen kompakten Quader mit rationalen Eckpunkten)
\section{Bedingte Wahrscheinlichkeit, Unabh\"angigkeit}
Gegenseitige Beeinflussung unter zuf\"alligen Ereignissen
\subsection{Bedingte Wahrscheinlichkeit}
Statt $P=P(\boldsymbol{\cdot})$ betrachte $P(\boldsymbol{\cdot}|B): \F \to [0,1]$ (bedingtes Wahrscheinlichkeitsma\ss{} unter der Bedingung $B$)\\
$P(B|B)=1, P(A|B=c_b*P(A), c_b\ge 1, A \subset B \in \F)$\\
Aus der Schule: $P_B(A)$\\
Ereignisse m\"ussen nNICHT zeitlich nacheinander erfolgen!\\
Viasualisierung am Baumdiagramm: TODO\\
\textbf{Definition:} f\"ur $P(B)>0$ definiert\\
\[\P(A|B)\overset{\Delta}{=}\frac{P(A \cap B)}{P(B)}\]
Wahrscheinlichkeitsraum $(\Omega, \F, P(\boldsymbol{\cdot}| B))$ erf\"ullt alle Axiome.\\
\textbf{Interpretationen:}
\begin{enumerate}
	\item subjektiv: pers\"onliche EInsch\"atzung des Beobachters bez\"uglich des Eintretens von $A$ nachdem er \"uber das Eintreten von $B$ informiert wurde
	\item frequentistisch: Bruchteil der F\"alle, in denen das Ereignis $A$ eintritt, bezogen auf die F\"alle, in denen das Ereignis $B$ auftritt\\
		$P(A|B)=\frac{n_{(A\cap B/n)}}{n_B/n}$
\end{enumerate}
Unabh\"angigkeit zweier Ereignisse $A_1,A_2$ gegeb Ereignis $A_3$: $P(A_1|A_3)*P(A_2|A_3)=P(A_1 \cap A_2 | A_3)$
\subsubsection{Gesetz der totalen Wahrscheinlichkeit und Satz von Bayes}
\textbf{Gesetz der totalen Wahrscheinlichkeit:}
\[P(A)=\sum_{i=1}^k P(A | B_i)P(B_i)\] mit disjunkten Mengen $B_i$ und $\cup_{i+1}^k B_i=\Omega$ (disjunkte Zerlegung von $\Omega$)\\
\textbf{Satz von Bayes}
\[P(B_j|A)=\frac{P(A|B_j)*P(B_j)}{P(A)}=\frac{P(A|B_j)*P(B_j)}{\sum_{i=1}^k P(A | B_i)P(B_i)}\]
\textbf{Multiplikationsatz:} $P(A|B)P(B)=P(A \cap B)=P(B|A)P(A)$
$P(A\cap B\cap C\cap D)=P(A|B\cap C\cap D)P(B|C\cap D)P(C|D)P(D)$ (usw.)
Unter Verwendung von Permutationen $\pi(n)$:\\
\[P(A_1_{\pi(1)})P(A_{\pi(2)|A_{\pi(1)}})P(A_{\pi(3)}|A_{\pi(2)}\cap A_{\pi(1)})\cdot \lots \cdot P(A_{\pi(k)}| A_{\pi(k-1)} \cap \ldots \cap A_{\pi(1)})\]
Wichtige Beispiele:\\
Wenn $A \subset B$ gilt $P(A \cap B)=P(A)$ sowie $P(B|A)=1$ aber nicht andersrum!
\subsubsection{Stochastische Unabh\"angigkeit}
\[P(A|B)=P(A),P(B|A)=P(B) \Leftrightarrow P(A \cap B)=P(A)P(B)\]
Es reicht eine der obigen 3 Gelichungen nachzuweisen!\\
Dann gilt f\"ur jede endliche Teilmenge $\emptyset \neq J \subset I$: $(\cap_{i\in I} A_i)=\pi_{i\in I} P(A_i)$
Eintreten von $B$ ver\"andert $A$ nicht.
Disjunkte Ereignisse sind wenn $P(A),P(B)\neq 0$ IMMER stochastisch abh\"angig
\subsubsection{Totale Wahrscheinlichkeit}
\[P(B)=P(B|A)P(A)+P(B|A^c)P(A^c)\]
Zum Pr\"ufen der Unabh\"angigkeit von $k$ Ereignissen m\"ussen $2^k-k-1$ Gleichungen erf\"ullt sein, wobei zur Pr\"ufung paarweiser Unabh\"angigkeit $\vect{k \\ 2}=\frac{k()k-1}{2}$ Gleichungen ausreichen (schw\"achere Bedingung)\\
Paarweise unabh\"angige Ereignisse sind nicht zwangsl\"aufig unabh\"anhig! Falls gilt
\[P(\cap_{i=1}^k A_i)=\Pi_{i=1}^k P(A_i)\]
bedeutet das nicht, dass die $k$ Ereignisse auch paarweise unabh\"angig sind.

\section{Zufallsvariablen und ihre Wahrscheinlichkeitsverteilung}
\subsection{Zufallsvariablen}
Wahrscheinlichkeitsraum $(\Omega,\F,P)$, Messraum $(\Omega',\F')$\\
\textbf{Zufallsvariable} Abbildung: $\mathrm{X}:\Omega \to \Omega'$, sodass f\"ur jedes Ereignis $A'\in \F$ im Bildraum ein Ereignis $A=\{\omega \in \Omega | \mathrm{X}(\omega) \in A'\} \in \F$ im Urbildraum existiert. Charakterisiert durch $(X,(\Omega,\F,P),(\Omega',\F'))$
Oft nennt man das Resultat der Abbildung auch $\mathcal{X}$. Verwendet werden Gro\ss{}buchstaben f\"ur die Abbildung  und Kleinbuchstaben f\"ur gew\"ohnliche Variablen.\\
Kurzschreibweise: $A=\{X \in A'\}=A=X^{-1}(A')$ obwohl $\mathrm{X}$ nicht bijektiv!\\
Mit Hilfe von Zufallsvariablen k\"onnen Experimente auf die zu untersuchende Fragestellung reduziert werden.\\
Man kann annehmen, das zu jedem Wahrscheinlichkeitsraum $\Omega,\F,P$ ein Ur-Wahrscheinlichkeitsraum $(\Omega_0,\F_0,P_0)$ existiert, welcher aber unbrauchbar f\"ur jegliche Modellierung w\"are.
\subsubsection{Charakterisierung:}
\begin{enumerate}
	\item Reelle Zufallsvariablen: Bildraum $\R$ mit Borelscher $\sigma$-Algebra $\B$\\
		WR $(\Omega,\F,P)$, MR $(\R,\B)$\\
		$\mathrm{X}: \Omega \to \R$ hei\ss{}t reele Zufallsvariable, falls $\{\omega \in \Omega | \mathrm{X}(\omega) \le x\}=\{X \le x\} \forall x \in \R$ gilt.\\
		Ergebnisse $\omega_i \in \Omega$, Realisierungen $x_i=\mathrm{X}(\omega_i)$\\
		Dise Anforderung wird f\"ur jede stetige Funktion $\mathrm{X}: \R^n \to \B^n$ erf\"ullt.
	\item Mehrdimensionale (reelle) Zufallsvariablen\\
		$\textbf{\mathrm{X}}: \Omega \to \R^n, n \in \N$ mit Borelscher $\sigma$-Algebra $\B^n$\\
		Zufallsvektor: $\textbf{\mathrm{X}}=\mat{\mathrm{X}_1,\ldots,\mathrm{X}_n}^\top$ mit\\
		$\mathrm{X}_i: \Omega \to \R$
	\item Komplexe Zufallsvariablen: Bildraum $\C$
		$\mathrm{X}: \Omega \to \C$ wenn $\Re{X}$,$\Im{X}$ reelle Zufallsvariablen sind.\\
		Zufallsvektor: $\textbf{\mathrm{X}}=\mat{\Re{X} & \Im{X}}$
\end{enumerate}
\subsection{Verteilung deiner Zufallsvariablen}
Bildma\ss{} $P_X(A')=P(\{X \in \A'\}) \forall A' \in \F'$ hei\ss{}t Verteilung der Zufallsvariablen und wird als das Bild von $P$ unter $X$ bezeichnet.
\subsubsection*{CDF}
Kumulative Verteilungsfunktion (KVF) von $X$, cumularive distribution funktion, cdf\\
\[F_\mathrm{X}(x)=P(\{X \le x\})\]
Eigenschaften:
\begin{itemize}
	\item $F_X(x)$ monoton wachsend
	\item $F_x(x)$ rechtsseitig stetig\\
		Notation mit gef\"ullten Kringeln bei Stetigkeit und ungef\"ullten Kringeln bei Unstetigkeit!
	\item $\Lim{x \to -\infty}F_X(x)=0$
	\item $\Lim{x \to + \infty} F_X(x)=1$
\end{itemize}
\textbf{Identit\"atsbildung:} $F_{X_{Id}}(x)=P(\{\omega \le x\})$ (Verteilungsfunktion von $P$)
\textbf{Arten von reellen Zufallsvariablen:}
diskret: Abbildung auf endlichem oder Abz\"ahlbarem Bildraum\\
stetig: TDODO\\
\subsubsection*{PMF}
Wahrscheinlichkeitsmassenfunktion (WMF) von X, probability mass function, pmf\\
\[p_\mathrm{X}(x)=P(\{X=x\})\]
Zusammenhang: $F_X(x)=\sum_{\xi \in \Omega':\xi \le x} p_\mathrm{X}(\xi)$\\
\textbf{Zufallsvariable X ist stetig, falls} ihre kumulative Verteilungsfunktion dargestellt werden kann als:
\[F_\mathrm{X}(x)=\int_{-\infty}^x f_\mathrm{X}(\xi) \mathrm{d} \xi\]
\subsubsection*{PDF}
Wahrscheinlichkeitsdichtefunktion (WDF) von X, probability density function, pdf\\
Absolut integrierbare Funktion $f_\mathrm{X}: \R \to [0,\infty[$\\
Wenn $F_\mathrm{X}$ stetig und bis auf endlich viele Stellen differenzierbar, dann ist $\mathrm{X}$ stetig und es gilt f\"ur differenzierbare $x$:
\[f_\mathrm{X}(x)=\frac{\mathrm{d} F_\mathrm{X}(x)}{\mathrm{d} x}\]
Wahrscheinlichkeit f\"ur reelle Zufallsvariablen bei Ereignis $\{X=x\}$:\\
\[P(\{X=x\})=\begin{cases} p_\mathrm{X}(x) \, \mathrm{X reell und diskret} \\ 0 \, \mathrm{X stetig} \end{cases}\]
Des weiteren existieren Zufallsvariablen mit sowohl diskretem, als auch stetigem Anteil. Dann existiert, werder eine Wahrscheinlichkeitsdichtefunktion, noch eine Wahrscheinlichkeitsmassefunktion im Sinn der Definition.\\
Wenn $D$ das Ereignis, dass eine Realisierung von $X$ dem diskreten Teil zuzurechnen ist und $S=D^c$ das Ereignis, dass sie dem stetigen Teil zuzurechnen ist, dann kann man schreiben:\\
\[\mathrm{X}=\begin{cases} X_D , \mathrm{Falls D eintritt} \\ X_S , \mathrm{Falls S eintritt}\end{cases}\]
Erweiterte kommulative Verteilungsfunktion von X:\\
\[F_X(x)=P(D)\sum_{\xi \le x:p_{X_D}(\xi)\neq 0} p_{x_D}(\xi) + P(D^c) \int_{-\infty}^x f_{X_{S}} (\xi) \mathrm{d} \xi\]
Normiertheitsbedingung auch hier erf\"ullt!\\
\textbf{Methodik zur einfachen Darstellung des Wahrscheinlichkeitsma\ss{}es von Zufallsvariablen mit diskretem und stetigem Anteil:}\\
Darstellung einer st\"uckweise konstaten Funktion $g(x)$ als summe gewichteter und verschobener Einheitssprungfunktionen $u(x)=\begin{cases} 1 , x \le 0 \\ 0 , x < 0 \end{cases}$ m\"oglich.\\
	Sprungfunktion im klassischen Sinne nicht differenzierbar! Aber da
	\[u(x)=\int_{-\infty}^x \dirac (\xi) \mathrm{d} \xi\]
	Dirac-Funktion als Ableitung der Einheitssprungfunktion: $\dirac (x)=\frac{\mathrm{d} u(x)}{\mathrm{d} x}$\\
	Beispiel: kommulative Verteilungsfunktion sowie Wahrscheinlichkeitsdichte einer diskreten Zufallsvariable X:\\
	\[p_{\mathrm{X}}=\begin{cases} \frac{1}{4} , \, x \in \{\frac{1}{8},\frac{3}{8},\frac{5}{8},\frac{7}{8}\} \end{cases}\]
	\[F_{\mathrm{X}}=\frac{1}{4}(u(x-\frac{1}{8})+u(x-\frac{3}{8})+u(x-\frac{5}{8})+u(x-\frac{7}{8}))\]
	\[f_{\mathrm{X}}=\frac{1}{4}(\dirac(x-\frac{1}{8})+\dirac(x-\frac{3}{8})+\dirac(x-\frac{5}{8})+\dirac{x-\frac{7}{8}})]]\]
% ======================================================================
% End
% ======================================================================
\end{document}
