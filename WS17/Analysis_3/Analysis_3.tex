% % % % % % % % % % % % % % % % % % % % % % % % % % % % % % % % % % % % % % % %
% LaTeX4EI Example for Cheat Sheets
%
% @encode:      UTF-8, tabwidth = 4, newline = LF
% @author:      LaTeX4EI
% % % % % % % % % % % % % % % % % % % % % % % % % % % % % % % % % % % % % % % %


% ======================================================================
% Document Settings
% ======================================================================

% possible options: color/nocolor, english/german, threecolumn
% default: color, english
\documentclass[nocolor,german]{latex4ei/latex4ei_sheet}

% set document information
\title{Analysis 3\\Zusammenfassung}
\author{Philipp van Kempen}                % optional, delete if unchanged
\myemail{philipp.van-kempen@tum.de}        % optional, delete if unchanged

	\newcommand{\Lim}[1]{\raisebox{0.5ex}{\scalebox{0.8}{$\displaystyle \lim_{#1    }\;$}}}
	
	\newcommand{\iftsymbol}{\mbox{\setlength{\unitlength}{0.1em}\begin{picture}(25,4)\put(3,3){\circle*{4}}\put(4,3){\line(1,0){13}}\put(19,3){\circle{4}}\end{picture}}}

	% Define Transformations
	\DeclareMathOperator{\T}{\overset{\scriptscriptstyle\mathcal{}}{\ftsymbol}}						% General Transform
	%\DeclareMathOperator{\FT}{\overset{\scriptscriptstyle\mathcal{F}}{\ftsymbol}}					% Fourier Transform
	%\DeclareMathOperator{\LT}{\overset{\scriptscriptstyle\mathcal{L}}{\ftsymbol}}					% Laplace Transform
	%\DeclareMathOperator{\DFT}{\overset{\scriptscriptstyle\mathcal{DF}}{\ftsymbol}}					% Discrete Fourier Transform
	%\DeclareMathOperator{\DTFT}{\overset{\scriptscriptstyle\mathcal{DTF}}{\ftsymbol}}				% Discrete Time Fourier Transform
	%\DeclareMathOperator{\ZT}{\overset{\scriptscriptstyle\mathcal{Z}}{\ftsymbol}}					% Z-Transform


	% Define inverse Transformations
	\DeclareMathOperator{\IT}{\overset{\scriptscriptstyle\mathcal{}}{\iftsymbol}}					% General Transform
	\DeclareMathOperator{\IFT}{\overset{\scriptscriptstyle\mathcal{F}}{\iftsymbol}}					% Fourier Transform
	\DeclareMathOperator{\ILT}{\overset{\scriptscriptstyle\mathcal{L}}{\iftsymbol}}					% Laplace Transform
	\DeclareMathOperator{\IDFT}{\overset{\scriptscriptstyle\mathcal{DF}}{\iftsymbol}}				% Discrete Fourier Transform
	\DeclareMathOperator{\IDTFT}{\overset{\scriptscriptstyle\mathcal{DTF}}{\iftsymbol}}				% Discrete Time Fourier Transform
	\DeclareMathOperator{\IZT}{\overset{\scriptscriptstyle\mathcal{Z}}{\iftsymbol}}					% Z-Transform

% DOCUMENT_BEGIN ===============================================================
\begin{document}

\maketitle      % requires ./img/Logo.pdf

\section*{Sonstiges}
\subsection{Notationen}
Vergleich zweier Funktionen: $f(x) \equiv \utilde{f}(x) $
\subsection*{Itentit\"aten}
	\begin{itemize}
        \item Frequenz $f=\frac{1}{T}$ in Hz, $f=\frac{\omega}{2 \pi}$
        \item $\overline{e^{i k}}=e^{- i k}$        \item komplexe Exponentialfunktion ist $2 \pi$-periodisch
        \item Stammfunktion der komplexen Exponentialfunktion wie in $\R$ allerdings gilt:\\
                $\forall m \in \Z: e^{2 \pi i m}=1 \Rightarrow \int_0^1 e^{2 \pi i m \tau} d \tau=0 \forall m \in \Z \setminus \{0\}$\\
				WICHTIG
        \item Eulersche Formel:\\
                $e^{i x}=\cos{x}+i \sin{x}$\\
                $\Rightarrow e^{-i x}=\cos{x}-i \sin{x}$\\
                mit $\cos{-x}=\cos{x}$ (gerade)\\
                und $\sin{-x}=-\sin{x}$ (ungerade)\\
				$\cos{x}=\frac{e^{i x}+e^{- i x}}{2}$\\
				$\sin{x}=\frac{e^{i x} - e^{- i x}}{2 i}$
        \item $i^2=-1$, $\frac{1}{i}=-i$
		\item $\cos{2 \pi k}={(1)}^k$\\
			$\cos{\pi k}={(-1)}^k$\\
			$\sin{\frac{\pi k}{2}}+sin{\frac{3 \pi k}{2}}=0$
		\item Hyperbolische Funktionen\\
			Definitionsbereich:?\\
			$\cosh{x}=\frac{e^x+e^{-x}}{2}$\\
			$\sinh{x}=\frac{e^x-e^{-x}}{2}$
	\end{itemize}
\subsection{Grundlagen}
	\begin{itemize}
		\item Fortsetzbarkeit in einer Unstetigkeitsstelle ist gegeben, falls der Funktionswert an Stelle $a$ nicht existiert, allerdings der rechtsseitige- und der linksseitige Limes gegen den selben Wert konvergieren! Hierbei hilft:\\
			$\Lim{t \to 0^+} f(t) = \Lim{t\ to 0^-} f(-t)$
	\end{itemize}
	
\subsection*{Beispiele}
\begin{itemize}
	\item S\"agezahn\\
		Skizze: TODO\\
		$f(t)=\frac{1}{2}(\pi - t), 2 \pi-$periodisch auf $0<t<2 \pi, f(0)=0$\\
		$S_f(t)=\sum_{n=1}^\infty \frac{1}{2} \sin{n}$\\
		Koeffizienten ebenfalls in jeder Formelsammlung\\
		S\"agezahn ist st\"uckweise $C^1$ (sogar $C^\infty$)\\
		allgemein: $f(t)=\frac{T}{2}-t$ auf $(0 < t < T)$
	\item Stammfunktion einer positiven Funktion, kann nicht periodisch sein!
	\item Nebenergebnis $\frac{\pi^2}{6}=1+\frac{1}{2^2}+\frac{1}{3^2}+\ldots$
	\item Treppenfunktionen lassen sich vereinfacht integrieren, wenn man das Integral an den Sprungstellen in eine Summe von Integralen splittet und ausnutzt, dass $f'(t)=0$ da $f(t)=\mathrm{const}$. Letzter Sprung muss wieder zum 1. Wert gelangen.
	\item Auswertung periodischer Funktionen: $f(6\pi)=f(6\pi-3 \cdot 2 \pi)\neq f(2\pi)$ da f\"r $2 \pi$ nicht definiert.
	\item Optimaler Tiefpass-Filter (GRAFIK) w\"urde alle Samples ben\"otigen. W\"ahle $d_k=1$ f\"ur zum Beispiel $k<5$ wenn $d_k$ Fourier-Koeffizient von $g$ und dem gefiltertem Signal $f \ast g$
\end{itemize}
\section{Orthogonalreihen, Integraltransformationen}
\subsection{Fourier-Reihen}
\textbf{Ziel:} Beschreibung eines periodischen Signals durch \"Uberlagerung harmonischer Schwingungen\\
\subsubsection{Grundlagen}
Funktionen $f: \mathbb{R} \to \mathbb{C}$ mit
\begin{itemize}
        \item $f$ ist \textbf{$T$-periodisch} $\Leftrightarrow f(t+T)=f(t) \forall t \in \R$
        \item $f$ ist \textbf{st\"uckweise stetig}, falls in jedem $t \in \R$ der \textbf{rechtsseitige Grenzwert} $f(t_+)=\Lim{\tau down t} f(\tau) \in \C$ und der \textbf{linksseitige Grenzwert} $f(t_-)=\Lim{\tau up t}$ existieren und in jedem beschr\"anktem Intervall $(a,b) \subset \C$ $f$ \textbf{stetig} ist, \textbf{bis auf endlich viele Punkte (Sprungstellen)}  
\end{itemize}
Periode $T$, Kreisfrequenz $\omega=\frac{2 \pi}{T}$\\
\textbf{Darstellungen:}\\
\begin{itemize}
        \item komplex:
                \[S_f(t)=\Lim{N \to \infty} \sum_{n=-N}^N c_n e^{i n \omega t}\]
				Mit Schwingungen $e^{i n \omega t}$ und Frequenzen $0, (0 Hz),\omega (\frac{1}{T}),2 \omega (frac{2}{T}),\ldots$\\
				Andere Schreibweise der Summe: $\sum_{k \in \Z}$
        \item sin-cos:
                \[S_f(t)=\frac{a_0}{2}+\sum{n=1}^{\infty}(a_n \cos (n \omega t)+ b_n \sin (n \omega t))\]
\end{itemize}
\textbf{Tipps:}\\
\begin{itemize}
	\item Integral splitten bei Betrag!
	\item Integralgrenzen an Intervall, nicht an $T$ anpassen!
	\item W\"ahlt man eine doppelt so gro\ss{}e periode, wird das $w$ dementsprechend kleiner, sodass die Koeffizienten sich \"andern. Allerdings sind die Fourierreihen \"aquivalent durch die verlangsamung der Schwinungen.
	\item Betrachte Symmetrieeingenschaften:
		\begin{itemize}
			\item gerade: $b_k=0 \forall k$
			\item ungerade $a_k=0 \forall k$
			\item weder noch $\exists a_k,b_k \neq 0$
			\item $f$ ungerade $\Rightarrow f \cdot \cos$ ungerade, $f \cdot \sin$ gerade
			\item $F$ gerade $\Rightarrow f \cdot \cos$ gerade, $f \cdot \sin$ ungerade
		\end{itemize}
\end{itemize}
\textbf{Formeln f\"ur Fourier-Koeffizienten:}\\
Fourier-Koeffizienten enthalten Amplituden und Phaseninformationen\\
komplex ($k \in \Z$): \[c_k=\frac{1}{T}\int_0^T f(t) e^{-i k \omega t} d t = \int_0^1 f(\tau T) e^{-2 \pi i k t} d \tau \in \C\]
Liegt der Mittelwert der zu untersuchenden Funktion \"uber eine Periodein der $x/t$-Achse, so ist $c_0$ stets gleich $0$\\
trigonometrisch ($k \in \N_0$ bzw. $k \in \N$):
$a_0=\frac{2}{T} \int f(t) d t$
\[a_k=\frac{2}{T}\int_0^T f(t) \cos (k \omega t) d t = 2 \int_0^1 f(t \tau) \cos (2 \pi k \tau) d \tau\]
\[b_k=\frac{2}{T}\int_0^T f(t) \sin (k \omega t) d t = 2 \int_0^1 f(t \tau) \sin (2 \pi k \tau) d \tau\]
Achtung: Teils ist Integrationsbereich $(-\pi,\pi)$ sinnvoller um Fallunterscheidung zu ersparen.\\
Ist $f$ reelwertig, dann folgt $a_k,b_k \in \R$ aber nicht $c_k \in \R$\\
\begin{tabular}{|l|l|l|}
        Art & \textbf{Fourieranalyse} & \textbf{Fouriersynthese} \\ 
		geg. & $f$ & ${(c_k)}_{k\in \Z}$ \\ 
		ges. & ${(c_k)}_{k\in \Z}$ & $S_f$ \\ 
        Notation & $\T$ & $\iftsymbol$ \\
\end{tabular}\\
Wobei ${(c_k)}_{k=0,\pm 1,\ldots}$ Spektralwerte hei\ss{}en\\
$S_f$ und $f$ stimmen bis auf Unstetigkeiten \"uberein! In Unstetigkeiten wird der Mittelwert des rechtsseitigen und linksseitigen Grenzwertes verwendet. (Mittelwerteigenschaft)\\
Eine Linearkombination von stetigen Funktionen ist stetig.\\
\textbf{Cauchy Hauptwert:} \[\sum_{n=-\infty}^{\infty}\ldots=\Lim{N \to \infty} \sum_{n=-N}^N \ldots\] TODO Fehler (symm. Grenzwert)\\
\textbf{Fundamentalbeziehungen:}
\begin{itemize}
        \item Orthogonalist\"tsrelation (OR)\\
                $k,n \in \Z$:\\
			$\frac{1}{T} \int_0^T e^{i n \omega t} e^{-i k \omega t} d t=\int_0^1 e^{2 \pi i (n-k) \tau} d \tau=\begin{cases} 0, (n\neq k)\\1, (n=k)\end{cases}=\delta_{n k}$\\
                mit Kronecker-Symbol $\delta_{i j}$
        \item Umrechnungsformeln (UR)
                \begin{itemize}
                        \item $c_k \to a_k,b_k$:\\
								$a_0=2 c_0$\\
                                $a_k=c_k+c_{-k}$ ($k \in \N_0$)\\
                                $b_k=\i(c_k-c){-k}$ ($k \in \N$)
                        \item $a_k,b_k \to c_k$:\\
                                $c_0=\frac{a_0}{2}$\\
                                $c_k=\frac{a_k-\i b_k}{2}$ ($k \in \N$)\\
                                $c_{-k}=\frac{a_k+i b_k}{2}$ ($k \in \N$)
                        \item falls $f$ $\R$-wertig:\\
                                $a_k=2 \Re{c_k}$ ($k \in \N_0$)\\
                                $b_k=2 \Im{c_k}$ ($k \in \N$)\\
								$c_{-k}=\overline{c_k}=\Re{c_k}-i \Im{c_k}$ ($k \in \Z$)
				\end{itemize}
Bei den Formeln f\"ur $c_k,a_k,b_k$ kann man $\int_0^T \ldots d t$ ersetzen durch $\int_a^{a+T} \ldots d t$\\
\end{itemize}
% TODO: replace \int_a^b f(x) dx with \int_a^b \! f(x) \, \mathrm{d}x

Ist $f$ ein \textbf{$T$-periodisches trigonometrisches Polynom}, also:\\
$f(t)=\sum_{n=-M}^N d_n e^{i n \omega t}$, so gilt:\\
\[c_k=\begin{cases} d_k, (-M \le k \le N) \\ 0 , \mathrm{sonst} \end{cases}\]
	$\Rightarrow$ F\"ur trigonometrische Polynome $f$ gilt $S_f=f$ (Fourierreihe=Funktion)\\
\textbf{Vorgehensweise:}\\
Falls $f(t)$ Summe von Einzeltermen:\\
\begin{enumerate}
	\item Bestimme kleinste Periode f\"ur Summanden $\Rightarrow$ bestimme kleinste geimeinsame Periode\\
		(konstante Funktion $f(t)=c$ periodisch f\"ur jede Periode $T$)\\
	\item Kreisfrequenz $\omega$ bestimmen
	\item Bestimmung der Koeffizienten:
		\begin{itemize}
			\item Mit Formeln: TODO
			\item Ablesen aus Euler-Formel:\\
				$\cos{t+\alpha}=\frac{1}{2}e^{i (t+\alpha)}+\frac{1}{2} e^{- i (t+\alpha)}$\\
				$\sin{b t}=-\frac{i}{2} e^{i b t}+\frac{i}{2} e^{-i b t}$\\
				$\Rightarrow$ $\frac{e^{i \alpha}}{2} e^{i t}$,$\ldots$ durch Auftrennen\\
				$c_0=$Konstanter Term\\
				$c_k=$ Term vor $e^{i k \omega t}$ (positiv und negativ)\\
				Weitere $c_k=0$?
		\end{itemize}
	\item Umrechnung:\\
		\begin{itemize}
			\item Mit Additionstheoremen $\cos{t+\alpha}$ auf Form $a_1 \cos{t} + b_1 \sin{t}$ bringen\\
				Umformung in eine $T$-periodische cos-sin-Reihe. Koeffizienten ablesen.
			\item Umrechnungsformeln (besser wenn $c_k$ bekannt) Andere $a_k,b_k=0$?
		\end{itemize}
\end{enumerate}
Achtung beim Einsetzen von Werten au\ss{}erhalb des Definitionsbreiches, wie z. B. $f(-t)$! Hier muss die Periode so oft addiert bzw subtrahiert werden, bis man nur noch im Definitionsbereich von $f$ liegt. Au\ss{}erdem drehen sich beim Einsetzen von negativem $t$ die Intervallgrenzen, sodass aus $[0,2 \pi)$ dann $(0,2 \pi]$ werden w\"urde.\\
\textbf{Zusammenh\"ange zwischen Funktion und Fourierkoeffizienten $c_k$:}\\
\begin{enumerate}
	\item Aufspalten: Berechnet man die Koeffizienten f\"ur jeden Sumanden einer Summe von Funktionen ist das Koeffizient der Summe, die Summe der Teilkoeffizienten
	\item Skalieren: Multiplikation der Funktion mit einem Faktor, bedeutet \"Anderung der Koeffizienten um den selben Faktor
	\item Spiegeln: Falls $c_k$ Koeffizient zu $f(t)$, so ist $c_{-k}$ Koeffizient zu $f(-t)$
\end{enumerate}
Falls $f(t)$ st\"uckweise stetig diff'bar so gilt $S_f(0)=f(0)$\\
Unterscheiden sich 2 Funktionen nur in endlich vielen Punkten, so ist ihre Fourier-Reihe dieselbe, da endlich viele Punkte f\"ur das Integral irrelevant!\\
\textbf{Fragen bez\"uglich deer Konvergenz:}
\textbf{Definition:} $f:\R\to\C$ hei\ss{}t st\"uckweise stetig differenzierbar, wenn $f$ auf jedem beschr\"anktem Intervall $(a,b) \subset \R$ bis auf h\"ochstens endlich viele Stellen stetig differenzierbar ist und zudem in jedem $t \in R f$ und $f'$ links- und rechts-seitige Grenzwerte $f(f-),f'(t-),f(t+),f'(t+)$ in $\C$ existieren!\\
Darum exklusiv, weil S\"agezahn da periodisch unendlich oft sprint. H\"aufung von Unstetigkeitsstellen NICHT erlaubt.\\
\begin{enumerate}
	\item Wo konvergiert $S_f(t)$ und wo gilt $S_f(t)=f(t)$?\\
		\textbf{Satz:} TODO f\"ur die Fourierreihe einer $T$-periodischen st\"uckweise stetig differenzierbaren Funktion ist:\\
		Punktweise Konvergenz $\forall t \in \R: S_f^N(t)=\Lim{N \to \infty} S_f^N \in \C TODO \forall t$ wobei $S_f^N(t)=\sum_{n+-N}^N c_n e^{i n \omega t}$ ($N$-te Partialsumme)
	\item Wo konvergiert $S_f$ gleichm\"a\ss{}ig gegen $f$?\\
		Gleichm\"a\ss{}ige Konvergenz von $S_f^N(t)$ gegen $f(t)$ auf jedem abgeschlossenem Intervall ohne Sprungstellen. Um so n\"aher man sich an den Sprungstellen befinden, um so h\"oher muss das $N$ werden, um eine Abweichung kleiner $\epsilon$ zu gew\"ahrleisten.
	\item Was passiert an den Sprungstellen?\\
		An jeder Sprungstelle $t$ gilt:
			\begin{itemize}
				\item Mittelwerteigenschaft: $S_f(t)=\frac{f(t-)+f(t+)}{2}$
				\item Gibbs-Ph\"anomen: GRAFIK, an erster \"Uberschwingung (Extrema) vor/nach Sprungstelle gilt $\Lim{N\to \infty}|\frac{S_f^N(t_N)-S_f^N(t)}{f(t+)-f(t)}|=\propto 1.18$ ($18$ Prozent)
			\end{itemize}
\end{enumerate}
\subsubsection*{RECHENREGELN}
	$f,g: \R \to \C$, st\"uckweise stetig, $T$-periodisch\\
	$f \T (c_k)$,$g \T (d_k)$
	\begin{itemize}
		\item Linearit\"at: $\forall \alpha,\beta \in \C: \alpha f + \beta g \T (\alpha c_k + \beta d_k)$
		\item Konjugation/Zeitumkehrung: $\overline{f} \T {(\overline{c_{-k}})}_{k \i \Z}$\\
			$f(-t)\T(c_{-k}){k\in\Z}$
		\item Streckung der Zeitskala: $\gamma>0$, $f(\gamma t) \T (c_k)$ (keine \"Anderung)\\
			$f(\gamma t)$ besitzt die Periode $\utilde{T}=\frac{T}{\gamma}$\\
		\item Verschiebung im Zeitbereich: $a \in \R, f(t+a) \T {(e^{i k \omega a c_k})}_{k \in\Z}$
		\item Verschiebung im Spektralbereich, Amplitudenmodulation: $e^{i n \omega t} f(t) \T {(c_{k-n})}_{k \in\Z}$, $n \in \Z$
		\item Symmetrien
			\begin{itemize}
				\item $f$ gerade: $f(\frac{T}{2}+t)=f(\frac{T}{2}-t) \Leftrightarrow f(t)=f(-t) \forall t$\\
					Graph spiegelsymmetrisch zur gerade $t=\frac{T}{2}$\\
					Prototyp: $\cos{k \omega t}$\\
					$\Rightarrow$ $c_k=c_{-k} \forall k \in \Z$, $a_k=\frac{4}{T} \int_0^{\frac{T}{2}} f(t)\cos{k \omega t} d t$ (Halbierung des Integrationsberechs und Verdopplung des Integralwerts)\\
					$b_k=0 (k \ge 1)$
				\item $f$ ungerade: $f(\frac{T}{2}+t)=-f(\frac{T}{2}-t) \Leftrightarrow f(t)=f(-t) \forall t$\\
					Graph punktsymmetrisch zu $(\frac{T}{2})$\\
					Prototyp: $\sin{k \omega t}$\\
					$\Rightarrow$ $c_k=c_{-k}$, $a_k=0$ ($k \in \N_0$)\\
					$b_k=\frac{4}{T} \int_0^{\frac{T}{2}} f(t) \sin{k \omega t}$ ($k \in \N$)
				\item $f$ $\frac{T}{2}-$periodisch: $f$ ist invariant gegen\"uber einer Zeitverschiebung um $\frac{T}{2}$. F\"ur Periode $T$ hat dann $f$ die Fourierkoeffizienten: $c_{2 k+1}=0$, $c_{2 k}=\frac{2}{T} \int_0^\frac{T}{2} f(t) e^{-2 k i \omega t} d t$, $a_{k+1}=0$, $b_{k+1}=0$\\
					$a_{2k}=\frac{4}{T} \int_0^{\frac{T}{2}} f(t) \cos{2 k \omega t} d t$\\
					$b_{2k}=\frac{4}{T} \int_0^{\frac{T}{2}} f(t) \sin{2 k \omega t} dt $\\
					Prototypen: $e^{- (2k) \omega t, WO IST J},\cos{2 k \omega t},\sin{2 k \omega t}$
				\item $f$ ohne $\frac{T}{2}-$periodischen Anteil: $f(\frac{T}{2}+t)=-f(t) \forall t$?\\
					$\Rightarrow c_{2k}=0,a_{2k}=0,b_{2k}=0$\\
					$c_{2k+1}=TODO,a_{2k+1}=TODO,b_{2k+1}=TODO$
			\end{itemize}
		\item Fourierreihe der Ableitung: $f$ stetig auf $\R$ und $f'$ st\"uckweise stetig\\
			$f'\T {(i k\omega c_k)}_{k\in\Z}$ aber $c_0=$?TODO
		\item Fourierreihe der Stammfunktion: $f$ st\"uckweise stetig\\
			$c_0=\frac{1}{T} \int_0^T f(t) d t=0$\\
			$F(t)=\int_0^t f(\tau) d \tau \T \begin{cases} \frac{1}{i k \omega} c_k , k \neq 0 \\ \frac{1}{T} \int_0^T t f(t) d t , (k=0)  \end{cases}$\\
				Bemerkung: $F$ $T$-periodisch erfordert $F(T)=F(0)$ und Freiheit vom Mittelwert.
		\item Ableitung bei Sprungstellen: $f$ st\"uckweise stetig mit $N$ Spr\"ungen bei $0 \le t_1 < t_2 < \cdots <t_N < =\mathrm{Periode}T$ zwischen den Sprungstellen ist $F\in \mathcal{C}^2$ ($N$ Sprungstellen pro Periode)
			Sprungh\"ohe bei $t_j$: $\Delta_j=f(t_{j}+)-f(t_{j}-)$\\
			$f'$ ist die Ableitung in Differenzierbarkeitspunkten. In den Sp\r"ungen ist $f'$ nicht definiert beziehungsweise auf einen beliebigen Wert gesetzt.\\
			Fourier-Koeffizienten: $f' \T {(i k \omega c_k - \frac{1}{T} \sum_{j=1}^N)\Delta_j e^{-ik\omega t_j}}_{k\in\Z}$\\
			Verallgemeinterte Ableitung mit \textbf{Dirac-Impuls} $\delta(t)=$Umpuls der St\"arke 1 bei $t=0$ mit den Eigenschaften $\int_{-\infty}^{\infty} \delta(t) d t=1$ und $\int_{-\infty}^{\infty}\delta(t) f(t) d t=f(0)$.\\
			Verschobener Dirac $\delta(t-a)$ analog mit $\int_{-\infty}^{\infty} \delta(t-a)f(t) d t=f(a)$\\
			Wenn $Df=$ Ableitung von $f$ inklusive Differenzieren der Sprungstellen:
			\[Df(t)=f'(t)+\sum_{j=1}^N \Delta_j \delta(t-t_j)\]
			mit Fourierkoeffizienten $Df \T {(i k \omega c_k)}_{k \in\Z}$\\
			(Wahl der Berechnung selbst entscheiden)
	\end{itemize}
	\textbf{Gr\"o\ss{}enordnung der Fourierkoeffizienten:}\\
	$f$ st\"uckweise setig und stetig differenzierbar bis auf endlich viele Sprungstellen pro Periode. Die Fourierkoeffizienten $(\hat{c_k})$ seien beschr\"ankt ($|\hat{c_k}\le C| \forall k$)\\
	Dann gibt es $M > 0$ mit $|\hat{c_k}| \le \frac{M}{|k|} \forall k \neq 0$

	\subsubsection{Periodische Faltung von $f,g$}
	\[(f \ast g)=\frac{1}{T}\int_0^T f(t-\tau)g(\tau)d \tau\] 
	Beispiele:\\
	$(1 \ast f)(t)=c_0 (f \T (c_k))$\\
	$(e^{i k \omega t} \ast f)=c_k e^{i k \omega t}$ (k-ter Summand der Fourierreihe von $f$)\\
	\textbf{Rechenregeln:}
	\begin{enumerate}
		\item Kommutativit\"at: $f \ast g=g \ast f$ (Beweis mit Substitution)
		\item Linearit\"at pro Faktor: $h \ast (\alpha f + \beta g)=\alpha (h \ast f)+\beta (h \ast g)$
		\item $f \ast g \T c_k d_k$ (Beweis mit Fubini)
		\item $f,g$ st\"uckweise stetig $\Rightarrow$ $f\ast g$ stetig. Integral wirkt gl\'attend, auch wenn $f,g$ sprungbehaftet
	\end{enumerate}
	\textbf{Gr\"o\ss{}enordnungen (Teil 2)}
	$f,f',\ldots,f^{(n-2)}$ st\"uckweise stetig, $f^{(n-1)}$ st\"uckweise stetig differenzierbar\\
	$\Rightarrow |c_k| \le \frac{M}{|k|^m}$ f\"ur $(k \neq 0)$\\
	Also: falls $f$ unstetig fallen die FK mit $\frac{M}{k}$, falls $f'$ unstetig mit $\frac{M}{k^2}$, wenn $f''$ unstetig mit $\frac{M}{k^3}$,\ldots\\
	\textbf{Anwendungen der Faltung:}
	\begin{enumerate}
		\item Interpretation: Faltung mit S\"agezahn\\
			$f(t)=(\frac{T}{\pi}s(\omega \dot) \ast f')(t)+c_0$\\
			mit Gleichspannungsoffset $c_0$ und Notation $f(\dot)$ f\"ur $f$\\
			Gleichheit gezeigt, da $f$ und der Ausdruck beide stetig sind und die Fourierkoeffizienten gleich.
		\item Frequenzgang von LTI-Systemen
			\textbf{LTI:} linear time invariant\\
			SYMBOL!\\
			Superposition: $\alpha x_1 + \beta x_2 \rightarrow \alpha y_1 + \beta y_2$ mit $x_i \rightarrow y_i$\\
			Zeitinvarianz: TODO\\
			Wenn Eingangssignal $T$-periodisch, ist auch $T$-periodisch.\\
			Ausgang $y(t)$ erf\"ullt lineare DGL mit $a_n \neq 0$:\\
			\[L[y](t):=a_n y^{(n)}+a_{n-1} y^{(n-1)}+\cdots+a_1 y'+a_0 y(t)=x(t)\]
			\textbf{Charakteristisches Polynom:}\\	
			$P(s)=a_n s^n + a_{n-1} s^{n-1} + \cdots + a_1 s + a_0$\\
			\textbf{Spezieller Input:} $x(t)=e^{i k \omega t}$\\
			L\"osungsansatz $y(t)=d_k e^{i k \omega t}$\\
			Mit Forderung $P(i k \omega)\neq 0 \forall k$ (keine Resonanz) gilt: $d_k=\frac{1}{P(i \omega k)} \forall k \in \Z$
	\end{enumerate}
	\textbf{Frequenzgang des LTI-Systems}
	$(d_k)_{k \in \Z}$ mit $d_k=\frac{1}{P(i \omega k)}$
	\textbf{$T$-periodische Impulsantwort des LTI-Systems} ist $H_T(t):=\sum_{n=-\infty}^{\infty} d_n e^{i n \omega t}$\\
	\"Ubertragungsgesetz: $y(t)=(h_T \ast x)(t)=\frac{1}{T} \int_{0}^T h_T(t-\tau)x(\tau)d \tau$\\
% document_end =================================================================
\end{document}
